% FreeDV-039 RADE presentation
% RADE presentation slides, intended for club level Ham presentation

\documentclass{beamer}

\usepackage{tikz}
\usetikzlibrary{calc,arrows,shapes,positioning,automata}
\usepackage{tkz-euclide}

\usetheme{CambridgeUS}

\title[RADE]{RADE - Machine Learning for Speech over HF Radio}
\institute[freedv.org]{freedv.org \and Supported by a grant from Amateur Radio Digital Communications}
\date[]{\today}

\begin{document}

% Tikz code used to support block diagrams
% credit: https://tex.stackexchange.com/questions/175969/block-diagrams-using-tikz

\tikzset{
block/.style = {draw, fill=white, rectangle, minimum height=3em, minimum width=3em},
tmp/.style  = {coordinate}, 
circ/.style= {draw, fill=white, circle, node distance=1cm, minimum size=0.6cm},
input/.style = {coordinate},
output/.style= {coordinate},
pinstyle/.style = {pin edge={to-,thin,black}}
}

% Title page frame
\begin{frame}
    \titlepage
\end{frame}

% Outline frame
\begin{frame}{Outline}
    \tableofcontents
\end{frame}

\section{Introduction}

\begin{frame}
\begin{description}
    \item[RADE] Radio Auto-encoder
\end{description}
\begin{itemize}
   \item Applying machine learning (ML) to send speech over HF radio 
   \item Combines traditional DSP and modern ML to encode and decode speech
   \item Connect a PC running RADE to your SSB radio
   \item 8 kHz audio bandwidth, high quality speech
   \item Works at low and high SNRs, handles multipath fading
\end{itemize}
\end{frame}

\begin{frame}
One of our crowd sourced examples, Texas to Australia, 25W, SNR 4 dB, deep fading
\href{http://freedv.org/davids-freedv-update-september-2024/}{\beamergotobutton{Link}}
\end{frame}

\begin{frame}
RADE is an outcome of the FreeDV Project ...
\begin{itemize}
    \item Open source HF digital voice for amateur radio
    \item Since 2023, funded by an ARDC grant
    \item 6 person Project Leadership Team
    \item Financial sponsor is the Software Freedom Conservancy
    \item Project Goal: a voice mode competitive with SSB at high and low SNRs
\end{itemize}
\end{frame}

\section{Design Versus Training}

\begin{frame}

\begin{itemize}
	\item Previously, we would design a system
	\item Figure out all the steps
	\item With ML the emphasis is on training a neural network
	\item Consider an AM receiver example
	\item Lets build the Detector using machine learning
\end{itemize}

\begin{tikzpicture}[auto, node distance=3cm,>=triangle 45,x=1.0cm,y=1.0cm,align=center,text width=2cm]

% AM receiver block diagram

\node [input] (rinput) {};
\node [block, below of=rinput, node distance=1.25cm,text width=2cm] (bpf) {Band Pass Filter};
\node [block, right of=bpf] (detector) {(Diode) Detector};
\node [block, right of=detector] (lpf) {Low Pass Filter};
\node [output, below of=lpf, node distance=1.25cm,text width=2cm] (routput) {};

\draw [->] node[above,text width=2cm] {Input RF} (rinput) -- (bpf);
\draw [->] (bpf) -- (detector);
\draw [->] (detector) -- (lpf);
\draw [->] (lpf) -- (routput) node[below,text width=1.5cm] {Output Speech};
\end{tikzpicture}

\end{frame}

\begin{frame}
\begin{itemize}
	\item Start with an untrained neural network
	\item Collect some training material
	\item Many examples of input and desired output
	\item Train the network so it matches desired output
\end{itemize}
\end{frame}

\begin{frame}
	TODO: slide with plots of network input and output
\end{frame}

\begin{frame}
\begin{itemize}
	\item Sometimes we don't know the best way to design something
	\item Real world problems are complex, prefect designs don't exist
	\item But we do have a good idea of what success looks (sounds) like
	\item So we just treat the system as a black box
	\item Show it examples of what we would like to see - and train
	\item ML has provided step changes in the quality of many algorithms
	\item Including speech synthesis and compression
\end{itemize}
\end{frame}

\section{RADE Design}

\begin{frame}
\begin{itemize}
	\item Jean-Marc Valin (and team) - compressing speech for Internet applications
    \item Idea: could it could be applied to noisy signal over radio channels?
    \item Jean-Marc developed a quick proof of concept
    \item We wrapped a modem around it to develop RADE
    \item Hams (and AREG SDRs) helped us crowd source the testing
    \item Mooneer Salem integrated RADE with the FreeDV GUI application
\end{itemize}
\end{frame}

\begin{frame}
\begin{tikzpicture}[auto, node distance=3cm,>=triangle 45,x=1.0cm,y=1.0cm,align=center,text width=2cm]

% Classical DSP System

\node [input] (rinput1) {};
\node [block, below of=rinput1, node distance=1.25cm,text width=2cm] (feature_ext) {Feature Extraction};
\node [block, right of=feature_ext] (quant) {Quantisation};
\node [block, right of=quant] (fec_enc) {FEC Encode};
\node [block, right of=fec_enc] (mod1) {Modulator};
\node [block, below of=mod1, node distance=2cm] (channel1) {Radio Channel};
\node [block, below of=channel1, node distance=2cm] (demod1) {Demodulator};
\node [block, left of=demod1] (fec_dec) {FEC Decode};
\node [block, left of=fec_dec] (dequant) {De-Quant};
\node [block, left of=dequant] (synth) {Speech Synth};
\node [output, below of=synth, node distance=1.25cm,text width=2cm] (routput1) {};

\draw [->] node[above,text width=2cm] {Input Speech} (rinput1) -- (feature_ext);
\draw [->] (feature_ext) -- (quant);
\draw [->] (quant) -- (fec_enc);
\draw [->] (fec_enc) -- (mod1);
\draw [->] (mod1) -- (channel1);
\draw [->] (channel1) -- (demod1);
\draw [->] (demod1) -- (fec_dec);
\draw [->] (fec_dec) -- (dequant);
\draw [->] (dequant) -- (synth);
\draw [->] (synth) -- (routput1) node[below,text width=1.5cm] {Output Speech};
\end{tikzpicture}
\end{frame}

\begin{frame}
\begin{tikzpicture}[auto, node distance=3cm,>=triangle 45,x=1.0cm,y=1.0cm,align=center,text width=2cm]

% RADE System

\node [input] (rinput) {};
\node [block, below of=rinput, node distance=1.25cm,text width=2cm] (feature_ext) {Feature Extraction};
\node [block, right of=feature_ext] (rade_enc) {\emph{RADE} Encoder};
\node [block, below of=rade_enc,node distance=2cm] (channel) {Radio Channel};
\node [block, below of=channel,node distance=2cm] (rade_dec) {\emph{RADE} Decoder};
\node [block, left of=rade_dec] (fargan) {Speech Synth};
\node [output, below of=fargan,node distance=1.25cm] (routput) {};

\draw [->] (rinput) node[above,text width=2cm] {Input Speech} -- (feature_ext);
\draw [->] (feature_ext) -- (rade_enc);
\draw [->] (rade_enc) -- (channel);
\draw [->] (channel) -- (rade_dec);
\draw [->] (rade_dec) -- (fargan);
\draw [->] (fargan) -- (routput) node[below,text width=1.5cm] {Output Speech};

\end{tikzpicture}
\end{frame}

\begin{frame}
\begin{itemize}
	\item Example of features - waterfall (aka spectrogram)
	\item Encoder takes waterfall and produces symbols we send over channel
	\item Decoder takes symbols and produces waterfall
	\item Which are fed to a ML synthesiser for high quality speech
	\item Trained with 200 hours of speech, corrupted by noise and the HF channel
	\item Classical DSP modem wrapped around that for synchronisation
\end{itemize}
\end{frame}

\section{Using RADE}

\begin{frame}
\begin{description}
    \item[Download] freedv.org 
    \item[Help] freedv.org - mailing list, QSO finder chat, Discord chat
    \item[QSO Finder] qso.freedv.org \href{http://qso.freedv.org}{\beamergotobutton{Link}}
    \item[PSK Reporter] \href{https://pskreporter.info/pskmap?preset&callsign=ZZZZZ&what=all&mode=FREEDV&timerange=86400&mapCenter=17.5,-8,2.6}{\beamergotobutton{Link}}    
\end{description}
\end{frame} 

\end{document}
